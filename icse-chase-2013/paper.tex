% This paper is written for CHASE 2013 and features the concept and
% creation of "Impact" the awareness tool.

\documentclass[conference]{IEEEtran}

\usepackage{graphicx}
\usepackage{amsmath}

\hyphenation{op-tical net-works semi-conduc-tor}

\begin{document}

%Paper title
\title{Impact: An Indirect Dependency Awareness Tool For Software Development}

%Author block
\author{\IEEEauthorblockN{Jordan Ell}
\IEEEauthorblockA{University of Victoria\\
Victoria, British Columbia \\ jell@uvic.ca}
\and
\IEEEauthorblockN{Daniela Damian}
\IEEEauthorblockA{University of Victoria\\
Victoria, British Columbia \\ danielad@cs.uvic.ca}}

\maketitle

% Paper abstract
\begin{abstract}
Awareness has been a largely studied field of awareness research within
software engineering. Many tools and techniques been been proposed and built
in order to provide software developers and stakeholders with a greater
sense of workspace and task awareness within their software projects. These
techniques and tools have been largely focused on detecting \textit{direct} 
conflicts which arise over a project's life time or have created 
an exploratory ground for stakeholders to use as a means of resolving self
discovered direct or \textit{indirect} conflicts. However, detecting 
and providing pertinent information regarding indirect conflicts has been
a largely ignored field of research partially due to its inherently largely
complexity than direct conflicts. Indirect conflicts arise from changes
in one software artefact affecting another. In this paper, we present
Impact, a new task awareness tool directly aimed at both detecting
and presenting indirect conflicts which arise inside of a software project.
We introduce previous indirect conflict awareness attempts, our design 
and implementation as well as describe Impact's potential through
two evaluation studies. 
\end{abstract}


\section{Introduction}

Awareness is characterized as "an understanding of the activities of others
which provides a context for one's own activities"~\cite{Dourish:1992:ACS}.
The study of awareness and its tools has become an important topic of
research in software engineering especially with the new importance of
distributed work and collaboration. Awareness is generally associated with
both technical and social dependencies that are created and evolve over
a software project's life time. The study of these dependencies has become
the primary focus of most awareness related research. Task awareness has
become the most prevalent field of research to understand how developers and
project managers cope with these technical and social dependencies.\\

Tools have been created to attempt to solve task awareness related issues
with moderate success~\cite{Xiang:2008:ERT, Biehl:2007:FVD, Sarma:2009:TIV, 
Khurana:2009:PFC}. However, these tools have been designed 
to solve task awareness related issues at the direct conflict level. 
Examples of direct task awareness include knowing when two or more 
developers are  editing same artefact, finding expert knowledge of a
particular file, and knowing which developers are working in which files.
Meanwhile, task awareness related issues at the indirect conflict level
continue to be an issue which is largely unsolved by most coordination
mechanisms~\cite{Khurana:2009:PFC}.\\

Previous interviews and surveys conducted with software developers show a 
consensus that developers of a software project view indirect task awareness 
as a high priority issue in their developement~\cite{Damian:2007:GSE, 
Halverson:2006:DTV, Begel:2010:CDE}. Examples of indirect task awareness
conflicts include have one's own code effect by another's source
code change and finding out who might be indirectly effect by one's
own code change. Indirect conflicts arising in source code are inherantly
difficult to resolve as most of the time, source code analysis must
be preformed in order to find the relationships harmed by these conflicts.
While some awareness tools have been created with these indirect conflicts
primarily in mind~\cite{Begel:2010:CDE, Trainer:2005:BGT}, most have only 
created an exploratory environment which is used by developers to
solve conflicts which may arise. These tools lack the ability to detect
indirect conflicts that arise and alert developers to their presence 
inside of the software system. Some tools have started to work directly
with solving indirect conflicts~\cite{Sarma:2007:TSA}, but continue
to be limited by their usefulness.\\ 

Despite software developer's need for indirect conflict awareness tools
and existing exploratory indirect conflict awareness tools, detecting
and alerting developers to arising indirect conflicts is still a major
problem in the field of task awareness. Impact, a web based awareness
tool, aims to solve this issue. In this paper, we describe Impact's
design and implementation in order to both detect newly created
indirect conflicts among software developers as well as alerting developers
to these conflicts. By leveraging technical relationships inherent of 
software projects with method call graphs as well as detecting changes
to these technical relationship through software configuration management
(SCM) systems, Impact is able to detect indirect conflicts as well as
alert developers involved in such conflicts in task awareness.\\

The rest of this paper is organized as follows. First, we begin by discussing
similar task awareness related tools which have partially solved the
issues presented by this paper and how their workings can be applied to 
Impact. In the following section we describe the design and implementation
of Impact as an awareness tool. We then discuss a preliminary evaluation of
Impact followed by a discussion of future work and conclusions.\\



\section{Related Work}



\section{Impact}

\subsection{Design}

\subsection{Implementation}



\section{Evaluation}



\section{Results}



\section{Conclusion and Future Work}



\bibliographystyle{IEEEtran}
\bibliography{paper}

%End of paper
\end{document}