% This paper is written for CHASE 2013 and features the concept and
% creation of "Impact" the awareness tool.

\documentclass[conference]{IEEEtran}

\usepackage{graphicx}
\usepackage{amsmath}

\hyphenation{op-tical net-works semi-conduc-tor}

\begin{document}

%Paper title
\title{Impact: An Indirect Dependency Awareness Tool For Software Development}

%Author block
\author{\IEEEauthorblockN{Jordan Ell}
\IEEEauthorblockA{University of Victoria\\
Victoria, British Columbia \\ jell@uvic.ca}
\and
\IEEEauthorblockN{Daniela Damian}
\IEEEauthorblockA{University of Victoria\\
Victoria, British Columbia \\ danielad@cs.uvic.ca}}

\maketitle

% Paper abstract
\begin{abstract}
Awareness has been a largely studied field of awareness research within
software engineering. Many tools and techniques been been proposed and built
in order to provide software developers and stakeholders with a greater
sense of workspace and task awareness within their software projects. These
techniques and tools have been largely focused on detecting \textit{direct} 
conflicts which arise over a project's life time or have created 
an exploratory ground for stakeholders to use as a means of resolving self
discovered direct or \textit{indirect} conflicts. However, detecting 
and providing pertinent information regarding indirect conflicts has been
a largely ignored field of research partially due to its inherently largely
complexity than direct conflicts. Indirect conflicts arise from changes
in one software artifact affecting another. In this paper, we present
Impact, a new task awareness tool directly aimed at both detecting
and presenting indirect conflicts which arise inside of a software project.
We introduce previous indirect conflict awareness attempts, our design 
and implementation as well as describe Impact's potential through
two evaluation studies. 
\end{abstract}


\section{Introduction}
Awareness is characterized as "an understanding of the activities of others
which provides a context for one's own activities"~\cite{Dourish:1992:ACS}.
The study of awareness and its tools has become an important topic of
research in software engineering especially with the new importance of
distributed work and collaboration. Awareness is generally associated with
both technical and social dependencies that are created and evolve over
a software project's life time. The study of these dependencies has become
the primary focus of most awareness related research. Task awareness has
become the most prevalent field of research to understand how developers and
project managers cope with these technical and social dependencies.\\

Tools have been created to attempt to solve task awareness related issues
with moderate success~\cite{Xiang:2008:ERT, Biehl:2007:FVD, Sarma:2009:TIV, 
Khurana:2009:PFC}. However, these tools have been designed 
to solve task awareness related issues at the direct conflict level. 
Examples of direct task awareness include knowing when two or more 
developers are  editing same artifact, finding expert knowledge of a
particular file, and knowing which developers are working in which files.
Meanwhile, task awareness related issues at the indirect conflict level
continue to be an issue which is largely unsolved by most coordination
mechanisms~\cite{Khurana:2009:PFC}.\\

Previous interviews and surveys conducted with software developers show a 
consensus that developers of a software project view indirect task awareness 
as a high priority issue in their development~\cite{Damian:2007:GSE, 
Halverson:2006:DTV, Begel:2010:CDE}. Examples of indirect task awareness
conflicts include have one's own code effect by another's source
code change and finding out who might be indirectly effect by one's
own code change. Indirect conflicts arising in source code are inherently
difficult to resolve as most of the time, source code analysis must
be preformed in order to find the relationships harmed by these conflicts.
While some awareness tools have been created with these indirect conflicts
primarily in mind~\cite{Begel:2010:CDE, Trainer:2005:BGT}, most have only 
created an exploratory environment which is used by developers to
solve conflicts which may arise. These tools lack the ability to detect
indirect conflicts that arise and alert developers to their presence 
inside of the software system. Some tools have started to work directly
with solving indirect conflicts~\cite{Sarma:2007:TSA}, but continue
to be limited by their usefulness.\\ 

Despite software developer's need for indirect conflict awareness tools
and existing exploratory indirect conflict awareness tools, detecting
and alerting developers to arising indirect conflicts is still a major
problem in the field of task awareness. Impact, a web based awareness
tool, aims to solve this issue. In this paper, we describe Impact's
design and implementation in order to both detect newly created
indirect conflicts among software developers as well as alerting developers
to these conflicts. By leveraging technical relationships inherent of 
software projects with method call graphs as well as detecting changes
to these technical relationship through software configuration management
(SCM) systems, Impact is able to detect indirect conflicts as well as
alert developers involved in such conflicts in task awareness.\\

The rest of this paper is organized as follows. First, we begin by discussing
similar task awareness related tools which have partially solved the
issues presented by this paper and how their workings can be applied to 
Impact. In the following section we describe the design and implementation
of Impact as an awareness tool. We then discuss a preliminary evaluation of
Impact followed by a discussion of future work and conclusions.\\


\section{Related Work}
Although there is an abundance of Awareness tools developed in research
today, only a handful have made an attempt to examine indirect conflicts.
Here, we will outline three of the forefront projects in indirect conflicts
and how these projects have influenced the decision making process in
the design and implementation of Impact.\\

We first start with both Codebook~\cite{Begel:2010:CDE} and 
Ariadne~\cite{Trainer:2005:BGT}. These projects produce an exploratory
environment for developers to handle indirect conflicts. Exploratory
pertains to the ability to solve self determined conflicts. Meaning that
once a developer discovers a conflict, they can use the tool as a type of
lookup table to solve their issue. Codebook is a type of social developer
network that is relating developers to source code, issue repositories and
other social media while Ariadne only looks at source code for developer
to source code association. In this, developers become
owners of source code artifacts by querying the source code repository
for ownership information. Both projects also use directed graphs
in order to relate technical artifacts to each other. This creates technical
dependencies within the source code which are used for indirect
conflicts. These projects make use of method call graphs in order to 
determine which methods invoke others. This is the basis for their 
linkage of source code artifacts. While these projects can be great tools 
for solving indirect conflicts which may arise, they lack the ability to
detect conflicts on their own.\\

A serious attempt at both detecting and informing developers of
indirect conflicts is the tool Palantir~\cite{Sarma:2007:TSA}. Palantir
monitors developers activities in files with regards to class signatures.
Once a developer changes the signature of a class by modifying changes
in name, parameters, return values of public methods etc., any workspace
of other developers which is using that file will be notified. Palantir utilizes
a push-based event model~\cite{Fitzpatrick:2002:SPA} which seems to be
a favored collection system among Awareness tools. Sarma et al. also
develops a generic design for future indirect conflict awareness tools. 
However, where Palantir falls short is in its collection and distribution
mechanisms. Palantir only considers "outside" appearance of technical
objects, being their return types, parameters, etc., and only delivers
detected conflicts to developers who are presently viewing or editing
the indirect object.\\

Impact is designed to handle the holes in the aforementioned projects.
Impact will focus on the detection of indirect conflicts at an internal level
of technical objects as opposed from object signatures and the distribution
of these conflicts to all appropriate developers regardless of their current
workspace activities. Impact is also designed around the successes these
projects have had in the past with directed technical artifact graphs as 
well as elements of collection, ownership and distribution functionality.\\


\section{Impact}
This section will proceed to give an outlined detail of Impact in both its
design and implementation. The design of Impact was created to be both
a generic construct which can be applied to other indirect conflict 
awareness tools while the implementation is specific to the technical
goals of Impact.

\subsection{Design}
Compared to tool design for direct conflicts, the major concern of 
indirect conflict tools is to relate technical objects to one another
with a "uses" mentality. To say that object 1 uses object 2 is to infer
a technical relationship between the two objects which can be used
in part to detect indirect conflict that arise from modifying object
2. This kind of relationship is modeled after a directed graph. Each
technical object and represented by node while each "uses"
relationship is represented by a directed edge. This representation
is used to indicate all indirect relationships within a software project.\\

While technical object relationships form the basis of indirect conflicts,
social communication is the ultimate goal of resolving such conflicts.
This being the case, developer ownership must be placed on the 
identified technical objects. With this ownership, we can now infer
relationships among developers based on their technical objects
"uses" relationship. Developer A who owns object 1 which uses 
object 2 owned by developer B may be notified by a change to
object 2's internal workings. Most, if not all, ownership information
of technical objects can be extracted from a project's source code
repository (CVS, Git, SVN, etc.).\\

Finally, the indirect conflict tool must be able to detect changes to
to the technical objects defined above. Two approaches have been 
proposed for change gathering techniques: real time and commit time.
Real time indicates that all changes made to source code will be
reported even without being committed to a project's central 
repository where as push time indicates only committed changes
will be subject to conflict analysis. We propose the use of push time
information gathering. Commit time avoids the issue of developers 
overwriting previous work or deleting modifications which would 
produce information for changes that no longer exist. However, the
trade off is that indirect conflicts must be committed before detected
which results in conflicts being apart of the system before being able
to be dealt with as opposed to catching conflicts before they happen.
Commit time also avoids the need for a client side application to 
monitor the development activity of project developers.\\

With this three step design of: (i) creating directed graphs of technical
objects, (ii) assigning ownership to those technical objects, and (iii)
detecting changes within commit time, we believe a wide variety of
indirect conflict awareness tools can be created or extended. The
implementation of Impact in the following section will follow these
three design guidelines.

\subsection{Implementation}
For Impact's implementation, we decided to focus on methods as our
selected technical objects to infer a directed graph from. The "uses" 
relationship described above form methods is method invocation.
Thus, in our constructed directed graph, methods represent nodes
and method invocations represent our directed edges. In order to 
construct this directed graph, abstract syntax trees (ASTs) are 
constructed from source files in the project. The ASTs allow us
to construct method call graphs~\cite{} from which the directed 
graphs can be constructed.\\

Once the directed graph is constructed, we must now assign
ownership to our technical objects (methods) as per our design.
To do this, we simply query the source code repository. In our case
we used Git as the source code repository, so the command "git blame"
is used for querying ownership information (most source code 
repositories have similar commands and functionality). This command 
gives authors of source code per line which can be used to assign
ownership to methods. If a method has 10 lines and developer A
has written 3 while developer B has written 7, then ownership is
assigned 30\% and 70\% respectively.\\

To detect changes to our technical objects (methods), we simply 
use a commit's \textit{diff} which is a representation of all changes
made inside a commit. We can use the lines changed in the diff to 
find methods that have been changed. This gives cause of potential
indirect conflicts. We now find all methods in our directed graphs
which invoke these changed methods. These are the final indirect
conflicts.\\

Now that the indirect conflicts have been found, we use the
ownership information of our technical objects to send alerts to
those developers involved in the indirect conflict. All owners
of methods which invoke those that have been changed are alerted
to the newly changed method. This can been seen in
Figure~\ref{}, the user interface of Impact. Here, in an RSS type
feed, the changing developer, time of change, changed method,
invoking methods, and commit message are all displayed. This 
list is auto updating as new commits are push to the central
git repository. Developers are now free to use this information
to be aware of and solve indirect conflicts which arise at the
method level.\\


\section{Evaluation}
To fully evaluate both the generic design of detecting and resolving
indirect conflicts as well as Impact, extensive testing and evaluation
must be preformed. However, we feel that a simple evaluation is
first needed to assess the foundation of our design and claims:
does Impact improve developer awareness with indirect conflicts
and help the resolving of such issues?.\\

We preformed two user case studies to address our question where
we gave Impact to two small development teams. Each team was
free to use Impact at their leisure during their development process,
after which interviews were conducted with leads developers from 
each development team. The interviews were conducted after each
team had used Impact for three weeks.\\

We asked lead developers to address two main concerns: do indirect
conflicts pose a threat at the method level, and did Impact help raise
awareness and promote quicker conflict resolution for indirect
conflicts. Our two interviews largely supported our initial claims of
indirect conflicts posing a serious threat to developers, especially
in medium to large teams or projects. Our interviews also pointed
out that method use can be a particularly large area for indirect
conflicts to arise. However, both interviewees also pointed out that
any technical object which is used as an interface to some data
construct or methodology, database access for instance, can be 
a large potential issue for indirect conflicts.  Interview response to
Impact was also largely positive, both interviewees said that Impact
helped raise awareness among their teams with what other developers
are doing as well as the influence it has on their own work. However,
both interviews showed Impact to have information overload. Both
interviews showed that while all method changes were being detected,
not all are alert worthy. One lead developer suggested to only alert
developers to indirect conflicts if the internal structure of a method
changes due to modification to input parameters or output parameters.
In other words, the boundaries of the technical objects (changing
how a parameter is used inside the method, modifying the return
result inside the method) seem to be more of interest than other 
internal workings.\\

These two studies have show that our design and approach to
detecting and alerting developers to indirect conflicts appear
to be on the correct path. Impact as a tool has laid the foundations
for future work in detecting indirect conflicts as well as alerting
developers although more thought must be given as to 
what constitutes a meaningful change inside our selected 
technical objects.\\


\section{Conclusion and Future Work}
In this paper, we have presented the issues that arise from indirect 
conflicts in present awareness tools. We have proposed a generic 
design for the future development of awareness tools in regards to
handling indirect conflicts. We have finally presented a prototype 
awareness tool Impact which was designed around our generic 
awareness approach. Impact was evaluated on a small scale, showing
its future potential as well as highlighting its current weaknesses.\\

In future work we are planning to conduct interviews and surveys
with software developers to confirm that indirect conflicts pose a
threat to their projects as well as discovering what constitutes a 
valid change inside of a given technical object. With these two
improvements, we plan to justify our generic design of indirect
conflict tools further as well as improve the ideas of detecting 
valid and significant changes to technical objects.\\

\bibliographystyle{IEEEtran}
\bibliography{paper}

%End of paper
\end{document}