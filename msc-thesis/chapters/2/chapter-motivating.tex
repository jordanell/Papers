\startchapter{Motivating Studies}
\label{chapter:motivating}

\newlength{\savedunitlength}
\setlength{\unitlength}{2em}

While the research problems have been briefly outlined in Chapter~\ref{chapter:introduction}, this chapter
will focus on the underlying studies which motivated the research of this thesis as well as give a more full
and rich description of the problem being solved.

In this chapter, two studies will be presented that I conducted which motivated, and gave insights into, 
the final research goals
of this thesis. The first study entitled ``Failure Inducing Developer Pairs''
(Section~\ref{study:pairs}), focuses on the prediction of
software failures through identifying indirect conflicts of developers linked by their software modules. 
This study found that certain pairs of developers when linked through indirect code changes are more prone
to software failures than others. The ideas of developer pairs linked in indirect conflicts will be
useful for the further development of indirect conflict tools as it shows that a human factor is present
and may be used to help resolve such issues.

The second study, ``Awareness with Impact'' ((Section~\ref{study:impact})), 
takes the notion of developer
pairs in indirect conflicts learned from Study 1, and adds in source code change detection in order to create
an awareness notification system for developers called \textit{Impact}. \textit{Impact} was designed to a developer
to any source code changes preformed by another developers when the two are linked in a technical dependency through
a developer pair. \textit{Impact} utilized a non-obtrusive RSS style feed for notifications for visualization. 
While \textit{Impact} showed some promise through its user evaluation, it ultimately suffered the fate of information
overload as was seen in other indirect conflict tools~\cite{Sarma:2007:TSA,Servant:2010:CPI,Trainer:2005:BGT}.

\input chapters/2/sec-developer-pairs
\input chapters/2/sec-impact

\setlength{\unitlength}{\savedunitlength}
