\section{Study 1: Failure Inducing Developer Pairs}
\label{study:pairs}

% Explain technical network research that already exists
Technical dependencies in a project can be used to predict success or failure in
builds or code changes~\cite{Pinzger:2008:DNP, Zimmermann:2008:PDU}. However, most 
research in this area is based on identifying
central modules inside a large code base which are likely to cause software failures or
detecting frequently changed code that can be associated with previous failures
~\cite{Kim:2006:AIB}. 
These module-based methods also result in predictions 
at the file or binary level of software development as opposed to a code change level
and often lack the ability to provide recommendations for improved coordination
other than test focus.

% Research question
With the power of technical dependencies in predicting software failures, the question I
investigated in this study was:

\begin{description}
        \item[RQ\namedlabel{itm:rq1-pairs}{RQ1-Pairs}] \textit{Is it possible to identify pairs 
        of developers whose technical dependencies in code changes statistically relate to bugs?}
\end{description}

% Explain our brief intended approach
This study explains the approach used to locate these pairs of developers in developer networks.
The process utilizes code changes and the call hierarchies effected to find patterns of developer 
relationships in successful and failed code changes. As it will be seen, I found 27 statistically
significant failure inducing developer pairs. These developer relationships can also be used
to promote the idea of leveraging socio-technical congruence, a measure of coordination compared
to technical dependencies amongst stakeholders, to provide coordination recommendations.

\subsection{Related Work}

Research has shown multiple reasons for software failures in both technical dependencies as well
as human or social dependencies in software development. On the technical side, studies have shown that
technical dependencies in software are often powerful predictors or errors in software as well as
in builds~\cite{Pinzger:2008:DNP, Zimmermann:2008:PDU, Hassan:4019574}. These technical dependencies
are often accompanied by data mining algorithms in order to set apart failure inducing dependencies
from non failing dependencies.

On the other side with the human factor, researchers have examined predicting build outcomes of software
using communication patterns from developers. Wolf et al.~\cite{5070503} used patterns of communication from between
pre-existing builds in order to predict later build outcomes. Naggappan et al.~\cite{nagappan08} showed that having a large
communication organizational difference between developers who worked on the same software module had
a negative influence on the quality of the software.

Studies have also combined both technical and social dependencies into a notion of socio-technical congruence.
Cataldo et al.~\cite{Cataldo:2006:ICR} have shown that this notion of socio-technical congruence can be leveraged to predict
and improve task completion times in a software projects.

These studies have mostly focused at a very high and abstract level of software development (builds and task completion).
Where they have fallen short is in fine grained analysis of software changes. This study is used to take the
pre existing knowledge of both technical and social dependencies and apply it to a source code change level of
granularity. Instead of large scale failures like long completion times or build failures, this study examines failures
at the bug level induced by each code change.


\subsection{Technical Approach}

\subsubsection{Extracting Technical Networks}
% End goal
The basis of this approach is to create a technical network of developers based on method ownership
and those methods' call hierarchies effected by code changes. These networks will provide
dependency edges between contributors caused by code changes which may be 
identified as possible failure inducing pairings (Figure~\ref{fig:network}). To achieve this goal,
developer owners of methods, method call hierarchies (technical
dependencies) and code change effects on these hierarchies must be identified.
This approach is described in detail by illustrating its application to mining the data in a Git
repository although it can be used with any software repository.

% Technical network figure
\begin{figure}[ht]
\centering
\includegraphics[width=0.9\textwidth]{figures/TecNetwork}
\caption{A technical network for a code change. Carl has changed method getX() which is being
called by Bret's method foo() as well as Daniel and Anne's method bar().\label{fig:network}}
\end{figure}

% Owners
To determine which developers own which methods at a given code change,
the Git repository is queried. Git stores developers of a file per line, which was used to extrapolate
a percentage of ownership given a method inside a file. If developer A has written 6/10 lines of 
method foo, then developer A owns 60\% of said method.

% Call Graph
A method call graph is then constructed to extract method call hierarchies in a project at a given code change. 
Unlike other approaches such as Bodden's et al.~\cite{Bodden:2003:HVJ} 
of using byte code and whole projects, call graphs are built directly from source code files
inside of a code change, 
which does not have the assumptions of being able to compile or have access to all project 
files. It is important to not require project compilation at each code change because it is
an expensive operation as well as code change effects may cause the project
to be unable to compile. Using source files also allowed an update to the call graph
with changed files as opposed to completely rebuilding at 
every code change. This creates a rolling call graph which 
is used to show method hierarchy at each code change inside a project opposed to
a static project view. As some method invocations may only be determined at run time, all
possible method invocations are considered for these types of method calls while constructing
the call graph.

% Changeset effects
The code change effect, if any, to the call hierarchy is now found. The Git
software repository is used to determine what changes were made to any given file inside a 
code change. Specifically, methods modified by a code change are searched for. The call graph 
is then used to determine which methods call those that have been changed, which
gives the code change technical dependencies.

%Resulting network
These procedures result in a technical network based on contributor method ownership 
inside a call hierarchy effected by a code change (Figure~\ref{fig:network} left hand side).
The network is then simplified by only using edges between developers, since I
am only interested in discovering the failure inducing edges between developers and not the 
methods themselves (Figure~\ref{fig:network} right hand side). This is the final technical 
network.

\subsubsection{Identifying Failure Inducing Developer Pairs}
To identify failure inducing developer pairs (edges) inside technical networks, 
edges in relation to discovered code change failures are now analysed. To determine whether a code change 
was a success or failure (introduce a software failure), the approach of
Sliwerski et al.~\cite{Sliwerski:2005:CIF} is used. The following steps are then taken:

\begin{enumerate}
\item Identify all possible edges from the technical networks.
\item For each edge, count occurrences in technical networks of failed code changes.
\item For each edge, count occurrences in technical networks of successful code changes.
\item Determine if the edge is related to success or failure.
\end{enumerate}

To determine an edge's relation to success or failure,  the value FI (failure
index) which represents the normalized chance of a code change failure in the presence
of the edge, is created. 

\begin{equation}
\text{FI} = \frac{ \text{edge}_{failed} / \text{total}_{failed}}{\text{edge}_{failed} / \text{total}_{failed} + \text{edge}_{success} / \text{total}_{success}}
\end{equation}

Developer pairs with the highest FI value are said to be failure inducing structures
inside a project. These edges are stored in Table~\ref{tab:ratio}. A Fisher Exact Value test 
is also preformed on edge appearance in successful and failed
code changes, and non-appearance in successful and failed code changes to only
consider statistically significant edges (Table ~\ref{tab:ratio}'s p-value). 


\subsection{Results}
To illustrate the use of the approach, I conducted a case study of
the Hibernate-ORM project, an open source Java 
application hosted on GitHub\footnote{https://github.com/} with issue tracking 
performed by Jira\footnote{http://www.atlassian.com/software/jira/overview}.

This project was chosen because the tool created only handles Java code and it is written in Java 
for all internal structures and control flow
and uses Git for version control. Hibernate-ORM also uses issue tracking software which 
is needed for determining code change success or failure~\cite{Sliwerski:2005:CIF}.

In Hibernate-ORM, 27 statistically significant failure inducing developer pairs (FI value of 0.5 or higher) 
were found out of a total of 46 statistically significant pairs that existed over the project's lifetime.
The pairings are ranked by their respective FI values (Table~\ref{tab:ratio}).

\begin{table}[h]
\begin{center}
\begin{tabular}{@{\hspace{.2cm}}ccc@{\hspace{.75cm}}c@{\hspace{.2cm}}c@{\hspace{.2cm}}}
\hline
Pair & Successful & Failed & FI & P-Value\\
\hline
(Daniel, Anne) & 0 & 14 & 1.0000 & 0.0001249 \\
(Carl, Bret)   & 1 & 12 & 0.9190 & 0.003468  \\
(Emily, Frank) & 1 &  9 & 0.8948 & 0.02165   \\
\hline
\end{tabular}
\end{center}
\caption{Top 3 failure inducing developer pairs found.\label{tab:ratio}}
\end{table}


\subsection{Conclusions of Study}
Technical dependencies are often used to predict software failures
in large software system~\cite{Pinzger:2008:DNP, Zimmermann:2008:PDU, Kim:2006:AIB}. 
This study has presented a method for detecting failure inducing pairs of developers inside
of technical networks based on code changes. These developer pairs can be used in the prediction
of future bugs as well as provide coordination recommendations for developers within a project.

This study however, did not consider the technical dependencies themselves to be the root cause of
the software failures. This study focused purely on developer ownership of software methods and
the dependencies between developers as the possible root cause of the failures. To study this root
cause futher, a study of indirect conflicts and their relationship to developer code ownership
will be conducted.