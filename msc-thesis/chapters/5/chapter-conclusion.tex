\startchapter{Conclusions}
\label{concl}

Awareness techniques have been proposed and studied to aid developer
understanding, efficiency, and quality of software produced. Some of these techniques have focused 
on either \textit{direct} or \textit{indirect conflicts} in order to prevent, detect, or resolve these conflicts as 
they arise from a result of source code changes. 

While the techniques and tools for direct conflicts have had large success, tools either proposed or studied for 
indirect conflicts have had common issues of information overload, false positives, scalability, information distribution 
and many others. These issues have ultimately led to poor developer interest in indirect conflict tools as well
as a failed adoption rate and partial failures of research.

To better understand these issues, this dissertation has focused on exploring the world of indirect 
conflicts through 4 studies. The first two studies focused on motivational circumstances which occur during 
the software development life cycle and cause indirect conflicts. Developer's interactions were studied in order to create 
a tool which can aid in the work-flows around indirect conflicts. The second two studies presented a deeper investigation 
as to why most indirect conflict tools fail to attract developer interest through exploring the root causes of indirect conflicts 
and how tools should be properly built to support developer work flows.

\section{Study 1}

Technical dependencies are often used to predict software failures
in large software system~\cite{Pinzger:2008:DNP, Zimmermann:2008:PDU, Kim:2006:AIB}. However, human interactions as tied
to software objects can also be used to predict software failures.
Study 1 presented a method for detecting failure inducing pairs of developers inside
of technical networks based on code changes. The methodology used was to assigned developers to the source code artifacts
which they authored. Next, dependencies were identified between source code artifacts, implying dependencies between
developers. Lastly, failure inducing dependencies between developers were identified through using methods to identify
bugs in source code throughout the lifetime of a project. These developer pairs were used in the prediction
of future bugs and to provide coordination recommendations for developers within a project. These bugs being predicted
directly correlate to indirect conflicts. Since the human dependencies being measured in Study 1 are directly tied to
software objects, we can say that the indirect conflicts being studied here are between developers themselves, while
the dependencies between the developers are found in the source code.

Study 1 however, did not consider the technical dependencies themselves to be the root cause of
the software failures. This study focused purely on developer ownership of software methods and
the dependencies between developers as the possible root cause of the failures.

Through the analysis of Study 1 as seen in Chapter 4, an important component of indirect conflicts are the developers
themselves and how their notions of other's work is perceived across a project. We can see from Study 1 results that
a developer Daniel may believe a source code artifact behaves in a particular way (from documentation or other forms
of communication for the artifact) so when that artifact changes its behavior, Daniel can become negatively impacted.
A human factor is present here as the person who makes the artifact change, say Anne, may not have communicated the
change to all other developers affected by the change. This is an important understanding that led directly into the
creation of Impact (Study 2) as a way to mitigate the negative affects across human developers as a result of 
source code changes.

\section{Study 2}

As a direct response to the findings and analysis of Study 1, Study 2 set out to address the human factors presented
through an indirect conflict tool. In Study 2, I proposed a generic design for the development of an awareness tool 
in regards to handling indirect conflicts through human factors. I presented a prototype awareness tool, \textit{Impact}, 
which was designed around the generic technical object approach. Impact could track which developers were responsible
for which artifacts of code as well as their dependencies to other developer's code artifacts. When a dependency to
another developer's code was changed, the developer was notified of the change in order to avoid any bugs that
may arise from the change. The delivery system used for Impact was that of an RSS type feed where developers could view
their notifications in a stream of messages through a web application. However, \textit{Impact} suffered from information 
overload, in that it had too many notifications sent to developers.

This failure from information overload was ultimately equivalent to the various other indirect conflict tools from
previous research (even those not addressing human developer factors). This issue of information overload is the key issue 
in preventing adopting and acceptance of indirect conflict tools from developers and ultimately leads to some research
component failures to understand what developers truly want from indirect conflict tools.As a proposed solution to information overload, 
the ideas of Meyer~\cite{Meyer:1988} on ``design by contract'' were presented. This methodology examines the 
post and pre conditions of software objects in order to reduce the number of source code changes that are analyzed by 
Impact in order to reduce information overload.

While the previous proposed solution was designed to fix information overload in Impact and potentially other indirect
conflict tools, it was decided that further investigation into indirect conflicts was needed to truly attempt a solution.
The results of Study 2, combined with the knowledge of previous research having similar issues of information overload
in indirect conflict tools, sparked an interest in studying indirect conflicts at a deeper level (Study 3) which could
be used to better understand causes and developer strategies in solving indirect conflicts.

\section{Study 3}

Indirect conflicts are a significant issue with real world development, however, many proposed techniques and tools to mitigate
losses in this realm have been unsuccessful in attracting major support from developers (as was seen in Study 2). 
In Study 3, I conducted a qualitative study involving
19 interviewed developers from 12 organizations as well as 78 surveyed developers. I provided characterization of indirect conflicts,
current developer work flow surrounding indirect conflicts, and what direction any future tools should follow in order to aid developers
in their work flow with indirect conflicts.

For the root causes of indirect conflicts, I found that indirect conflicts occur due to changing of a software object's 
contract and the lack of understanding of
the far reaching implications (through dependencies) of that change. I also found that indirect conflicts are more likely
to occur at the beginning of a development cycle when the code is quite unstable and that these scenarios can become
compounded in difficulty when more developers are present on a project. 

As per current developer work-flows regarding indirect conflicts, I found that developers have 3 major mitigation strategies
to avoid indirect conflicts: ``Design by Contract'', add and deprecate, and personal experience. For catching indirect conflicts,
I found that use case coverage through proper testing (both unit and integration) are currently thought to be the best
developers can achieve. And finally, for the future of indirect conflict tools, developers made it clear that they would
prefer an improvement to resolution methods for indirect conflicts such as automatic or aided development techniques.

As per the analysis of Study 3, I have shown the disjoint of current awareness understanding versus the practical
awareness needs found in industry. This disjoint, caused by the difference of academic and practical understanding in awareness needs,
ultimately lead to tools with information overload and false positive issues. The debate of prevention versus cure was presented along with
the industrial tendency towards curative measures. While a curative approach may be favored by developers, further research is needed
to fully assess the positive and negative influences of prevention versus cure for productivity and quality concerns. Finally, I have shown
the gap of analytical evolution tools needed for dependency identification and indirect conflicts. This gap, unless addressed, may prevent
future industrial adoption of tools produced by researchers for lack of fit in industrial needs. This gap directly resulted in the final
study of this Dissertation to be completed (Study 4).

\section{Study 4}

As a result of the gap in software evolution analysis shown in Study 3, I conducted a case study of 10 open source Java software 
projects in order to study their change trends surrounding
major releases as Study 3 had shown that indirect conflicts occur less at a major release and more so after a major release
or at the beginning of a development cycle. 

Through Study 4, I presented 9 major change trends which surround major releases in the open source
projects studied. The 4 change trends found in this study occurring before major releases are: added private classes, 
changed test method signatures, changed documentation, and removed test classes.
The 5 change trends found in this study occurring after major release are: added test methods, changed test classes, removed public methods, removed
private classes, and removed private methods.

These 9 change trends can be used in indirect conflict tools in order to identify a context for indirect conflicts.
For example, the trend of adding test methods was found to occur more after a major release than before. This trend along with
the knowledge that indirect conflicts occur more at the start of a new development cycle can be used to predict times in development
where indirect conflicts are likely to occur automatically. An automated system may see that more test methods are being added than
before in the past 4 weeks or so and be able to alert developers to an instability in the code which in turn means an elevated
risk of indirect conflicts.

The results of Study 4 directly supplemented those from Study 3 in that it addressed the gap of software evolution techniques
needed for indirect conflicts. However, this study was also quite limited and the issues of cross domain analysis techniques
presented in Study 3 are still cause for future research.

\section{Final Conclusions}

This dissertation has covered a wide range of interests all within indirect conflicts. I have shown how human factors can
be an important part of indirect conflicts and how pairs of developers can be found to be statistically related to indirect
conflicts bugs. I have shown how these human factors can be integrated into indirect conflict tools by using dependencies in 
authored source code artifacts. Ultimately however, these new human factors added into indirect conflict tools resulted in
similar failures of information overload as seen through many previous research tools in indirect conflicts. After these conclusions,
I presented a study which found root causes of indirect conflicts to be contract changes and an unawareness of those changes
implications, that developers use ``Design by Contract'', add and deprecate, and personal experience to avoid indirect conflicts,
and some suggestions for future indirect conflict tool development while stating where gaps in source code analysis should
be improved to improve the indirect conflict tools. Based on that gap in
software evolution analysis in indirect conflicts, I presented a method for finding contextual patterns which can relate to
indirect conflicts in order to aid future tool development for indirect conflicts.


Through the analysis in Chapter 4, I have shown the disjoint of current awareness understanding versus the practical
awareness needs found in industry. This disjoint, caused by the difference of academic and practical understanding in awareness needs,
ultimately led to tools with information overload and false positive issues. This was quite evident from my tool Impact as
presented in Study 2. Even when the tool is based around the statistical results of Study 1, the disjoint between what academics
determine developers want versus what they themselves want is quite obvious. The debate of prevention versus cure has been presented along with
the industrial tendency towards curative measures. However, while a curative approach may be favored by developers, further research is needed
to fully assess the positive and negative influences of prevention versus cure for productivity and quality concerns. Finally, I have shown
the gap of analytical evolution tools needed for dependency identification and indirect conflicts. This gap, unless addressed, may prevent
future industrial adoption of tools produced by researchers for lack of fit in industrial needs. While, I have shown a potential solution to address
a fraction of the analysis gap through contextual patterns of indirect conflicts, many other problems of the analysis gap remain problems
for future research.
