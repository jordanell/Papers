\section{Study 3: An Exploration of Indirect Conflicts}
\label{study:exploration}

I want to understand why it is so hard to tackle indirect conflicts, specifically through tool based solutions.
To do so, I take a step back and intend to obtain a broader view of indirect conflicts
with a large field study of practitioners. I investigate the root causes of information overload, false positives, and
scalability issues in regards to indirect conflicts to better understand why indirect conflict tools fail to achieve
the success of other domain tools. I determine: general events which 
cause indirect conflicts, when they occur, if compounding conditions exist, mitigation strategies developers
use, and what practitioners want from new tools. My research questions for this particular study are as follows:

\begin{description}
        \item[RQ1\namedlabel{itm:rq1}{RQ1}] \textit{What are the types, factors, and frequencies of indirect conflicts?}
        \item[RQ2\namedlabel{itm:rq2}{RQ2}] \textit{What mitigation techniques are used by developers in regards to indirect conflicts?}
        \item[RQ3\namedlabel{itm:rq3}{RQ3}] \textit{What do developers want from future indirect conflict tools?}
\end{description}

I interviewed 19 developers from across 12 commercial and open source projects, followed by a confirmatory questionnaire of 78 
developers, and 5 confirmatory interviews, in order to answer the aforementioned questions. My findings indicate that: 
indirect conflicts occur frequently and are likely caused by software contract changes and a lack of understanding,
developers tend to prefer to use detection and resolution processes or tools
over those of prevention, developers do not want awareness mechanisms which provide non actionable results, 
and there exists a gap in software evolution analytical tools from the reliance on static analysis resulting in missed
context of indirect conflicts.

\subsection{Related Work}
\label{sec:iced-related}

Many indirect conflict tools have been created, tested, and published. Sarma et al.~\cite{Sarma:2007:TSA} created Palantir,
which can both detect potential indirect conflicts, at the class signature level, and alert developers to these conflicts.
Palantir represented one of the first serious attempts at aiding developers towards indirect conflicts. Holmes et
al.~\cite{Holmes:2010:CAR} take it one step further with their tool YooHoo, by detecting fine grained source code changes,
such as method return type changes, and create a taxonomy for different types of changes and their proneness to cause
indirect conflicts. This tool and its taxonomy had a severely reduced false positive rate, however, the true positives
detected may already be detectable by current tools such as compilers and unit testing. The tool Ariadne~\cite{Trainer:2005:BGT}
creates an environment where developers can see how source code changes will affect other areas of a project at the
method level, using method call graphs, showing where indirect conflicts may occur. This type of exploratory design has
been used often in the visualization of indirect conflict tools, allowing developers a type of search area for their development
needs. Another indirect conflict tool, CASI~\cite{Servant:2010:CPI}, utilizes dependency slicing~\cite{Bajracharya:2009:SIS}
instead of method call graphs to provide an environment to see what areas of a project are being affected by a source code change.
Most of these tools have all shown to have the same common difficulties of scalability, false positives, and information overload,
which were explored in this study. My own tool
Impact!~\cite{Ell:2013} also suffered these same fates.

Since Cataldo et al.~\cite{Cataldo:2006:ICR} have shown that socio-technical congruence can be leveraged to improve task completion
times, many indirect conflict tools support the idea of a socio-technical congruence~\cite{Kwan:2011:ESC} in order to help
developers solve their indirect conflict issues through social means~\cite{Begel:2010:CDE}~\cite{Borici:2012:CHA}.
Begel et al.~\cite{Begel:2010:CDE} created Codebook, a type of social developer network related directly to source code
artifacts, which can be used to identify developers and expert knowledge of the code base. Borici et al.~\cite{Borici:2012:CHA}
created ProxiScientia which used technical dependencies between developers to create a network of coordination needs.
Socio-technical congruence however, is largely unproven in regards to its correlation to
software quality~\cite{Kwan:2011:SCE} and again the problems of scalability and information overload become a factor.

For developer interest in regards to software modifications, Kim~\cite{Kim:2011:ESA} found that developers wanted to know whose
recent code changes semantically interfere with their own code changes, and whether their code is impacted by a code change.
Kim found that developers are concerned with interfaces of objects and when those interfaces change. Kim also identified the same 
issues towards information overload through false positives with developers noting ``I get a big laundry list... I see the email and I delete it''.
Kim's field study does however fall short in actually creating a resolution for indirect conflicts, or finding new concerns of
developers which are not already detected by compilation or other static analysis tools. 
For impact awareness, DeSouza et. al.~\cite{deSouza:2008:ESS} found that developers use their personal knowledge of the
code base to determine the impact of their code changes on fellow developers, teams, and projects. However, DeSouza does not
study which types of changes (types, frequencies, compounding factors) developers should concern themselves with more in terms of using their personal knowledge
to stop indirect conflicts. DeSouza also does not study formal mitigation strategies, or resolutions of indirect conflicts directly.

This study is intended to fill the gap which has been left by aforementioned tools papers as well as the field study paper.
I will not only study why information overload, false positives, and scalability are such difficult problems to tackle
in indirect conflict tools, but I will also study how developers currently deal with indirect conflicts in practice through
their mitigation strategies, their largest concerns, and their ideas for future suggestions in regards to indirect conflicts.

\subsection{Methodology}
\label{sec:exp-meth}

I performed a mixed method study in three parts. First, a round of semi-structured interviews were conducted which 
addressed my 3 main research questions. Second, a questionnaire was conducted
which was used to confirm and test what was theorized from the interviews on a larger sample size as well as obtain
larger developer opinion of the subject. Third, 5 confirmatory interviews were conducted by re-interviewing original
interview participants to once again confirm my insights.
I used grounded theory techniques to analyze the information provided from all three data gathering stages.

\subsubsection{Interview Participants}

My interview participants came from a large breadth of both open and closed source software development companies
and projects, using both  agile and waterfall based methodology, and from a wide spectrum of organizations, as shown in Table~\ref{tab:demo}.
My participants were invited based on their direct involvement in the actual writing of software for
their respective companies or projects. These participants' software development experience ranged from 3-25 years of experience
with an average of 8 years of experience.
In addition to software development, some participants were also invited based on their experience with project management
at some capacity. See Table~\ref{tab:demo} for more demographic details.

\begin{table*}[tb!]
\begin{center}
\begin{tabular}{| p{2cm} | p{2cm} | p{2cm} | p{2cm} | p{2cm} | p{2cm} |}
\hline
Company & \# of Participants & Software Development Experience (years) & Development Process & Software Access & Current Language Focuses \\
\hline
\hline
Amazon & 2 & 5, 7 & Agile & Closed source & C++ \\ \hline
Exporq Oy & 1 & 8 & Agile & Closed source & Ruby, JavaScript \\ \hline
Fireworks Design & 1 & 6 & Agile & Closed source & JavaScript \\ \hline
Frost Tree Games & 1 & 4 & Agile & Closed source & C\# \\ \hline
GNOME & 1 & 13 & Agile & Open source & C \\ \hline
James Evans and Associates & 5 & 3, 3, 3, 4, 13 & Waterfall & Closed source & Oracle Forms \\ \hline
Kano Apps & 1 & 10 & Agile & Closed source & JavaScript, PHP \\ \hline
IBM & 2 & 5, 18 & Agile & Open and closed source & Java, JavaScript \\ \hline
Microsoft & 2 & 6, 10 & Agile & Closed source & C\# \\ \hline
Mozilla & 1 & 25 & Agile & Open source & C++, JavaScript \\ \hline
Ruboss & 1 & 5 & Agile & Closed source & JavaScript \\ \hline 
Subnet Solutions & 1 & 5 & Agile & Closed source & C++ \\ \hline
\end{tabular}
\end{center}
\caption{Demographic information of interview participants.\label{tab:demo}}
\end{table*}

\subsubsection{Interview Procedure}

Participants were invited to be interviewed by email and were sent a single reminder email one week
after the initial invitation if no response was made. I directly emailed 22 participants and  conducted
19 interviews. Interviews were conducted in person when possible and recorded for audio content only. When in person
interviews were not possible, one of Skype or Google Hangout was used with audio and video being recorded but only
audio being stored for future use. 

Interview participants first answered a number of demographic questions. I then asked them to describe various software development experiences regarding my three research questions.
Specifically, ten semi-structured topics directly related to my research questions guided my interview: 


\begin{itemize}
\item What tools are used for dependency tracking?
\item What processes are used for preventing indirect conflicts?
\item What artifact dependency levels are analyzed and where can conflicts arise?
\item How are software dependencies found?
\item Give examples of indirect conflicts from real world experiences.
\item How are indirect conflicts detected or found?
\item How are indirect conflict issues solved or dealt with?
\item Opinion of preemptive measures to prevent indirect conflicts.
\item What types of changes are worth a preemptive action?
\item Who is responsible for fixing or preventing indirect conflicts?
\end{itemize}

While each of the 10 topics had a number of starter questions, interviews 
largely became discussions of developer experience and opinion as opposed to direct answers to any specific question.
However, not all participants had strong opinions or any experience on every category mentioned. For these participants, answers 
to the specific categories were not required or pressed upon. I attribute any non answer by a participant to
either lack of knowledge in their current project pertaining to the category or lack of experience in terms of
being apart of any one software project for extended periods of time. 

\subsubsection{Questionnaire Participants}

My questionnaire respondents were different from my interviewees. I invited my questionnaire participants from a similar breadth of open and closed source software development 
companies and projects as the interviews participants with two main exceptions. The software organizations
that remained the same between interview and questionnaire were: Mozilla, The GNOME Project, Microsoft Corporation, 
Subnet Solutions, and Amazon.
Participants who took part in the round of interviews were not invited to the questionnaire but were asked to act as a point of contact for other 
developers in their team, project, or organization who may be interested
in completing the questionnaire. Further, two other groups of developers were asked to
participate as well, these being GitHub users as well as Apache Software Foundation (Apache) developers. The GitHub
users were selected based on large amounts of development activity on GitHub and the Apache developers
were selected based on their software development contributions on specific projects known to be used heavily
utilized by other organizations and projects.

\subsubsection{questionnaire Procedure}

questionnaire participants were invited to participate in the questionnaire by email. No reminder email was sent as the
questionnaire responses were not connected with the invitation email addresses and thus participants who did respond
could not be identified. I directly emailed 1300 participants and ended with 78 responses
giving a response rate of 6\%. I attribute the low response rate with: the questionnaires
were conducted during the months of July and August while many participants may be away from their regular positions.
and my GitHub and Apache participants could not be verified as to whether or not they actively support the
email addresses used in the invitations. In addition, the questionnaire was considered by some to be long and require
more development experience than may have been typical of some of those invited to participate.

The questionnaire I designed ~\footnote{http://thesegalgroup.org/people/jordan-ell/iced\_survey/}
was based on the insights I obtained from the round of interviews, and was intended to confirm some of these insights but also to broaden them to a larger sample size of developers who may have similar
or different opinions from those already acquired from the interviews. The questionnaire went through two rounds of
piloting. Each pilot round consisted of five participants, who were previously interviewed, completing the questionnaire
with feedback given at the end. Not only did this allow me to create a more polished questionnaire, but it also allowed 
the previously interviewed developers to examine the insights I developed. 

The question topics asked in the questionnaire were:

\begin{itemize}
\item What frequency do ICs occur at around different project milestones?
\item How does team size affect IC frequency?
\item What software change types do developers care about?
\item What processes are used for preventing ICs?
\item What tools are used for detecting ICs?
\item What tools are used for debugging ICs?
\end{itemize}


\subsubsection{Data Analysis}
To analyze both the interview and questionnaire data, I used grounded theory techniques as described by Corbin and Strauss~\cite{Corbin:1998:SP}.
Grounded theory is a qualitative research methodology that utilizes \textit{theoretical sampling} and
\textit{open coding} to create a theory ``grounded'' in the empirical data. For an exploratory study such as
mine, grounded theory is well suited because it involves starting from very broad and abstract type questions, and
making refinements along the way as the study progresses and hypotheses begin to take shape. Grounded theory involves
realigning the sampling criteria throughout the course of the study to ensure that participants are able to answer new
questions that have been formulated in regards to forming hypotheses. In my study being presented, data collected from
both interviews and questionnaires (when open ended questions were involved) was analyzed using open and axial coding. Open coding involves
assigning codes to what participants said at a low sentence level or abstractly at a paragraph or full answer level. These
codes were defined as the study progressed and different hypotheses began to grow. I finally use axial coding it order to
link my defined codes through categories of grounded theory such as context and consequences. 
In Section~\ref{sec:exp-eval}, I give a brief evaluation of my studying using 
3 criteria that are commonly used in evaluating grounded theory studies.

\subsubsection{Validation}

Following my data collection and analysis, I re-interviewed 5 of my initial interview participants
in order to validate my findings. I confirmed my findings as to whether or not they resonate with 
industry participants' opinions and experiences regarding indirect conflicts and as to their industrial 
applicability. Due to limited time constraints of the interviewed participants, I could only re-interview
five participants. Those that were re-interviewed came from the range of 5-10 years of software development
experience. Re-interviewed participants were given my 3 research questions along with results and main
discussion points, and asked open ended questions regarding their opinions and experiences to validate my 
findings. I also evaluated my grounded theory approach as per Corbin and Strauss'~\cite{Corbin:1998:SP}
list of criteria to evaluate quality and credibility. This evaluation can be seen in Section~\ref{sec:exp-eval}

\subsection{Results}
\label{sec:exp-results}

I now present my results of both the interviews and questionnaires conducted in regards to my 3 research questions
outlined in this chapter and Chapter 1. I restate each research question, followed by my quantitative and qualitative
results from which I draw my discussion to be seen in Chapter~\ref{chapter:discussion}. In each subsection, quantitative data
given refers to interviews conducted unless explicitly said otherwise.

\subsubsection{What are the types, factors, and frequencies of indirect conflicts?}

\bf{The most common occurrence of indirect conflict is when a software object's contract changes. The frequency of
indirect conflicts, while usually high, decreases as a stable point is reached in the development cycle. The frequency
of indirect conflicts is compounded by the number of developers working on a project.}

\normalfont{}

12 developers believe that a large contributing factor to the cause
of indirect conflicts comes from the changing of a software object's contract. Object contracts are, in a sense,
what a software object guarantees, meaning how the input, output, or how any aspect of the object is guaranteed
to work; made famous by Eiffel Software's~\footnote{http://www.eiffel.com/} ``Design by Contract''\texttrademark.
In light of object contracts, 14 developers gave examples of indirect conflicts they had experienced
which stemmed from not understanding the far reaching ramifications of a change being made to an object contract
towards the rest of the project. Of those 14, 3 dealt
with the changing of legacy code, with one developer saying ``legacy code does not change because developers
are afraid of the long range issues that may arise from that change''. Another developer, in regards to changing
object contacts stated ``there are no changes in the input or changes in the output, but the behavior is different''.
Developers also noted that the conflicts that do occur tend to be quite unique from each other and do not necessarily
follow common patterns.

In regards to object contract changes, 9 developers currently working with large scale database applications listed database
schemas as a large source of indirect conflicts while 5 developers that work on either software
library projects or are in test said that methods or functions were the root of their indirect conflict issues.
7 developers mentioned that indirect conflicts occur when a major update to an external project,
library, or service occurs with one developer noting ``their build never breaks, but it breaks ours''. Some
other notable indirect conflict artifacts were user interfaces in web development and full components in component
base game architecture.

11 developers explained that indirect conflicts occur ``all the time'' in
their development life cycle with a minimum occurrence of once a week, with more serious issues tending
to occur once a month. To confirm this, 64\% of questionnaire participants answered that indirect conflicts occur bi-weekly or more
frequent,
with the 25\% saying that weekly occurrences are most common (seen in Table~\ref{tab:often}). 5 questionnaire participants
also stated that the stage of the development cycle can greatly cause the frequency of indirect conflicts to
differ.

12 developers said that
when a project is in the early stages of development, indirect conflicts tend to occur far more frequently
than once a stable point is reached. Developers said ``At a stable point we decided we are not going to change
[this feature] anymore. We will only add new code instead of changing it.'' and ``the beginning of a project
changes a lot, especially with agile''. Questionnaire participants also added ``indirect conflicts after a release
depend on how well the project was built at first'', ``[indirect conflicts] tend to slow down a bit after a
major release, unless the next release is a major rework.'', and ``[indirect conflicts have] spikes during
large revamps or the implementation of cross-cutting new features.'' in order to confirm mu interview results.
Questionnaire participants also answered that
indirect conflicts are more likely to occur before the first major release rather than after at the daily
and weekly occurrence intervals as seen in Table~\ref{tab:often}.

\begin{table*}[tb!]
\begin{center}
\begin{tabular}{| p{2cm} | c | c | p{1.2cm} | c | p{1.3cm} | c | c |}
\hline
Occurrences & Daily & Weekly & Bi-Weekly & Monthly & Bi-Monthly & Yearly & Unknown \\
\hline
\hline
In general & 18\% & 25\% & 21\% & 16\% & 3\% & 5\% & 11\% \\ \hline
Early stages of a development & 32\% & 18\% & 4\% & 5\% & 0\% & 5\% & 36\% \\ \hline
Before the first release & 13\% & 29\% & 6\% & 8\% & 1\% & 3\% & 40\% \\ \hline
After the first release & 6\% & 18\% & 8\% & 18\% & 1\% & 5\% & 44\% \\ \hline
Late stages of development & 6\% & 5\% & 5\% & 18\% & 8\% & 12\% & 46\% \\ \hline
\end{tabular}
\end{center}
\caption{Results of questionnaire as to how often indirect conflicts occur, in terms of percentage
of questionnaire participants.\label{tab:often}}
\end{table*}

In terms of organizational structure, questionnaire participants answered that as a project becomes larger and more
developers are added, even to the point that multiple teams are formed, indirect conflicts become more likely to
occur. However, indirect conflicts still occur at a lower number of developers as well with even 43\% of developers
saying they are like to occur in single developer projects. This can be seen in Table~\ref{tab:env}.

\begin{table*}[tb!]
\begin{center}
\begin{tabular}{| p{3cm} | c | c | c | c | c |}
\hline
Environment of conflicts & Strongly Disagree & Disagree & Neutral & Agree & Strongly Agree \\
\hline
\hline
Developing alone (conflicts in own code) & 18\% & 20\% & 19\% & 24\% & 19\% \\ \hline
Developing in a team between 2 - 5 developers (inter-developers conflicts) & 3\% & 8\% & 22\% & 49\% & 18\% \\ \hline
Developing in a multi team environment (inter-team conflicts) & 1\% & 11\% & 14\% & 39\% & 35\% \\ \hline
\end{tabular}
\end{center}
\caption{Questionnaire results about development environments in which indirect conflicts are likely to occur, in terms of percentage
of questionnaire participants.\label{tab:env}}
\end{table*}

\subsubsection{What mitigation techniques are used by developers in regards to indirect conflicts?}

\bf{Three preventative techniques are used to mitigate indirect conflicts: design by contract, add and deprecate, and personal
experience. For catching indirect conflicts, developers use: testing (unit and integration) with proper use case coverage,
scheduled build processes, and static analysis tools built into IDEs. For resolving indirect conflicts, developers use: historical
error and information logs, compiler and debugger tools, as well as static analysis tools built into IDEs.}

\normalfont{}

In terms of preventative processes used for indirect conflicts, 3 major components were found.
First, design by contract is heavily used by interviewed developers as a means to avoid indirect conflicts or understand
when they are likely to occur. The use of design by contract here means that developers tend not to change an object's
contract when possible, and that an object's contract is used as a type of documentation towards awareness of the
software object. One developer stated that ``design by contract was invented to solve this problem and it does it
quite well'', while another noted that software object contracts do solve the problem in theory, but that doesn't
mean that problems don't still occur in practice.
Second, 21\% of developers mentioned that the add and deprecate
model is used to prevent indirect conflicts once
the project, feature, or component has reached a stable or mature point.
Add and deprecate meaning instead of editing code, the developer simply clones old code (if needed), and edits the clone
while slowly phasing out the out the old code in subsequent releases or ad needed. This allows other software to
use the older versions of software objects which remain unchanged, thus avoiding indirect conflicts.
Lastly, pure developer experience was mentioned with 7 developers mentioning that when planning code changes,
either a very experienced member of the project was involved in the planning and has duties to foresee any
indirect conflicts that may arise, or that developers must use their personal knowledge to predict where indirect
conflicts will occur while implementing.

Of the the 37 questionnaire participants who could give positive identification of preventative processes for indirect conflicts,
27\% said that individual knowledge of the code base and their impact of code change was used, 59\% mentioned some form of design
by contract or the testing of a methods contract, and 14\% said that add and
deprecate was used in their projects to avoid indirect conflicts. This is a confirmation of what was found in the interviews.

In regards to catching indirect conflicts, 13 developers mentioned forms of testing (unit, and integration)
as the major component of catching indirect conflict issues, subscribing to the idea of ``run the regression and integration
tests and just see if anything breaks''. The words ``use case coverage'' were constantly being used by developers
when expressing how proper unit and integration tests should be written. Developers expressed that with proper use case coverage, most if
not all indirect conflicts should be caught. In corroboration, 31\% of questionnaire participants said build processes (either nightly builds or building the project
themselves), and others mentioned code reviews while those dealing with a user interface mentioned user complaints from run
time testing. The questionnaire participants confirmed these results with 49\% mentioning forms of testing as the major tool used to
catch indirect conflicts, 33\% said build processes, while 31\% used work their IDE or IDE plug-ins to catch indirect conflicts.
Questionnaire participants also mentioned review process and personal expertise as factors of catching indirect conflicts.

Once an indirect conflict has occurred and developers need to resolve it, 14 developers said
they checked historical logs to help narrow down where the problem could originate from. Most developers had the mindset of
``Look at the change log and see what could possibly be breaking the feature.''. The log most commonly referred to was the source
code change log to see what code has been changed, followed by build failure or test failure logs to examine errors messages and get time
frames of when events occurred. Of the questionnaire participants, 23\% said they used native IDE tools and 21\% said they use
features of the language's compiler and debugger in order to solve indirect conflicts. Interestingly, only 13\% of developers
mentioned a form of communication with other developers in aid to solving these conflicts and only 4\% mentioned the reading
of formal documentation.

%Very important
Through the processes and tools of prevention, detection, and resolution of indirect conflicts, it is important to note that
most developers ascribe to the idea of ``I work until something breaks'', or taking a curative rather than preventative
approach. This means that while developers do have processes and tool for prevention, they would rather spend their time
at the detection and resolutions stages. One developer noted that preventative processes are ``too much of a burden''
while a project manager said ``[with preventative process] you will spend all you time reviewing instead of implementing''.

\subsubsection{What do developers want from future indirect conflict tools?}

\bf{Being that developers believe ``good enough'' solutions exist for preventing and detecting indirect conflicts
developers would rather see improvement to debugging processes. Developers are looking for automated debugging tools,
and better source code analysis tool for impact management and supporting cross team and cross sub-project projects.}

\normalfont{}

When asked about preventative tools, developers had major concerns that the amount of false positives
provided by the tool which may render the tool useless. Developers said ``this would
be a real challenge with the number of dependencies'', ``it depends on how
good the results are in regards to false positives'', and ``I only want to know if it will break me'', meaning that
developers seem to care mostly about negative impacts of code changes as opposed to all impacts in order to reduce
false positives and to keep preventative measures focused on real resulting issues as opposed to preventing potential
issues. Overall, developers had little interest in preventative tools or processes.

In terms of catching indirect conflicts, developers suggested that proper software development processes
are already in place to catch potential issues such as testing, code review,
individual knowledge, or static language analysis tools. Further, developers said that having strict change type monitoring
for indirect conflicts does not work due to the inherent complexities of indirect conflicts.
This can be seen confirmed in that questionnaire participants had very few cases in which they always wanted to be
alerted to a change type as seen in Table~\ref{tab:pre}. Only method signature changes caused 68\% of questionnaire participants
to want to be always notified as they have a high chance to break the code. This is opposed to other change types listed in
Table~\ref{tab:pre} which have 27\% or lower of developers always wanting to be notified. This again showcases the
complexity of indirect conflicts through change types which may or may not negatively affect a project.

\begin{table*}[tb!]
\begin{center}
\begin{tabular}{| p{3cm} | c | c | c | c | c |}
\hline
Code change type &Never  & Occasionally & Most Times & Always & I Don't Care \\
\hline
\hline
Method signature change & 5\% & 8\% & 12\% & 68\% & 7\% \\ \hline
Pre-condition change & 5\% & 27\% & 37\% & 23\% & 7\% \\ \hline
Main algorithm change & 11\% & 45\% & 19\% & 15\% & 11\% \\ \hline
User interface change & 12\% & 32\% & 20\% & 27\% & 9\% \\ \hline
Run time change & 13\% & 29\% & 25\% & 20\% & 12\% \\ \hline
Post-condition change & 7\% & 28\% & 32\% & 23\% & 11\% \\ \hline
\end{tabular}
\end{center}
\caption{Questionnaire results about source code changes that developers deem notification worthy, in terms of percentage
of questionnaire participants.\label{tab:pre}}
\end{table*}

When asked about curative tools, developers could only suggest that resolution times be decreased by different means.
Interview and questionnaire participants suggested the following improvements to curative tools:

\begin{itemize}
        \item Aid in debugging by finding which recent code changes are breaking a particular area of code or a test.
        \item Automatically write new tests to compensate for changes.
        \item IDE plug-ins to show how current changes will affect other components of the current project.
        \item Analysis of library releases to show how upgrading to a new release will affect your project.
        \item Built in language features to either the source code architecture (i.e. Eiffel or 
                                Java Modeling Language~\footnote{http://www.eecs.ucf.edu/~leavens/JML//index.shtml}) or the compile 
                                time tools to display warning messages towards potential issues.
        \item A code review time tool which allows deeper analysis of a new patch to the project allowing the reviewer to see potential 
                                indirect conflicts before actually merging the code in.
        \item A tool which is non-obtrusive and integrates into their preexisting development styles without them having to take extra steps.
\end{itemize}

\subsection{Evaluation}
\label{sec:exp-eval}

From the re-interviewed participants, I found extremely positive feedback regarding both my results and major discussion
points. Participants often had new stories and experiences to share once they had heard the results of this study which
confirmed the findings and often were quite shocked to hear the results as they did not usually think about their actions
as such but then realized the results held true to their daily actions for better or worse.

As per grounded theory research, Corbin and Strauss list ten criteria to evaluate quality and credibility~\cite{Corbin:1998:SP}.
I have chosen three of these criteria and explain how I fulfill them.

{\bfseries Fit.} ``Do the findings fit/resonate with the professional for whom the research was intended and the participants?'' This
criterion is used to verify the correctness of my findings and to ensure they resonate and fit with participant opinion. It is also
required that the results are generalizable to all participants but not so much as to dilute meaning. To ensure fit, during interviews
after participants gave their own opinions on a topic, I presented them with previous participant opinions and asked them to comment
on and potentially agree with what the majority has been on the topic. Often the developers own opinions already matched those of
the majority before them and did not necessarily have to directly verify it themselves.

As added insurance, I conducted 5 post results interviews with developers to once again confirm my results, and discussions. These
procedures can be seen in Section~\ref{sec:exp-meth}.

To ensure the correctness of the results, I also linked all findings in Section~\ref{sec:exp-results} to either a majority of agreeing
responses on a topic or to a large amount of direct quotes presented by participants.

{\bfseries Applicability or Usefulness.} ``Do the findings offer new insights? Can they be used to develop policy or change practice?''
Although my results may not be entirely novel or even surprising, the combination of said results allowed me to discover a
the disjoint between theoretical awareness and practical awareness regarding indirect conflict tools as well as provide more
insight into the debate of prevention versus cure in software development (as to be seen in Chapter~\ref{chapter:discussion}). 
Given how many indirect conflict tools are left with the same common
issues, I believe that these findings will help researchers focus on what developers want and need moving into the future more than has
been possible in the past. These finding set a course of action for where effort should be spent in research to better benefit industry.

10 of the 78 questionnaire participants sent direct responses to me asking for any results of the research to be sent directly to
them in order to improve their indirect conflict work flows. 7 of the 19 participants interviewed expressed interest concerning any possible
tools or plans for tools inspired by this research as well. The combination of research relatability and direct industry interest 
in my results help us fulfill this criterion.

{\bfseries Variation.} ``Has variation been built into the findings?'' Variation shows that an event is complex and that any findings
made accurately demonstrate that complexity. Since interviewed participants came from such a diverse set of organizations, 
software language knowledge, and experience the variation naturally reflected the complexity. Often in interviews and questionnaires, participants
expressed unique situations that did not fully meet my generalized findings or on going theories. In these cases, I worked in the specific
cases which were presented as boundary cases and can be seen in Section~\ref{sec:exp-results} as some unique findings or highly
specialized cases. These cases add to the variation
to show how the complexity of the situation also resides in a significant number of unique boundary situations as well as the complexity
in the generalized theories and findings.

\subsection{Threats to Validity}
Because of the exploratory nature of this work, I chose Grounded Theory as my research method, which has some implications regarding the
limitations and threats to validity of this study. While I achieved high saturation regarding the topics focused on in this study, other
populations studied from outside of this studies participants may add new insights into the pool of findings. As a result of this, findings
from this study may not relate to everyone or generalize to outside of the group of participants involved in this study.

As per my interview participants, I attempted to cover as great a breadth as possible of software developers in their practices, companies,
project, languages, etc. However, it would be extremely difficult to conduct a completely thorough study of indirect conflicts in practice
as there exist too many demographic options for participants. This same applies for questionnaire participants. This being said, questionnaire 
or interview results could become heavily biased towards a developers good or bad experiences with any given project.

To counter act this, my confirmatory interviews found high agreement for several of my findings. Therefore, I believe that
I have uncovered valuable insight regarding indirect conflicts in practice.

\subsection{Conclusions of Study}

In this study, I have explored indirect conflicts by examining their root causes, their current mitigation strategies, and how developers
wish to handle indirect conflicts in the present in future. To achieve these results, I interviewed 19 developers from across 12 commercial 
and open source projects, followed by a confirmatory questionnaire of 78 developers, and 5 confirmatory interviews.

My findings have indicated that:
indirect conflicts occur frequently and are likely caused by software contract changes;
while design by contract, add and deprecate, and personal experience help prevent indirect conflicts,
developers tend to prefer to use detection and resolution processes or tools
over those of prevention, and developers want indirect conflict tools to focus on automatic debugging and better source code analysis.

My result analysis has indicated that: developers do not want awareness mechanisms which provide non actionable results; developers
would rather focus on curing existing problems rather than preventing potential issues;
and that there exists a gap in software evolution analytical tools between what is available and what practitioners need.
