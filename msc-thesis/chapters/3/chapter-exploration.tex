\startchapter{Exploring Indirect Conflicts}
\label{chapter:exploration}

As Software Configuration Management (SCM) has grown over the years, the maturity and norm of parallel 
development has become the standard development process instead of the exception. With this parallel development
comes the need for larger awareness among developers to have ``an understanding of the activities of others
which provides a context for one's own activities''~\cite{Dourish:1992:ACS}. This added awareness
mitigates some downsides of parallel development which include the cost of conflict prevention and resolution. However,
empirical evidence shows that these mitigated losses continue to appear quite frequently and can prove to be a significant
and time-consuming chore for developers~\cite{Perry:2001:PCL}.

Through the two previous studies, I have shown that developers linked indirectly in source code changes can statistically
related to software failures. In the attempts of mitigating these loses through added awareness, I implemented
an indirect conflict tool called \textit{Impact}. However, \textit{Impact} ultimately suffered from information
overload as seen in its evaluation which was caused by false positives and scalability of the tool.

While other indirect conflict tools have shown potential from developer studies, some of the same problems continue
to arise throughout most, if not all tools. The most prevalent issue is that of information
overload and false positives. Through case studies, developers have noted that current indirect conflict tools provide too many 
false positive results leading to information overload and the tool eventually being
ignored~\cite{Sarma:2007:TSA, Servant:2010:CPI}. A second primary issue is that of dependency identification and
tracking. Many different dependencies have been proposed and used in indirect conflict tools such as method 
invocation~\cite{Trainer:2005:BGT}, and class signatures~\cite{Sarma:2007:TSA} with varying success, but the 
identification of failure inducing changes, other than those which are already identifiable by other means such
as compilers, and unit tests, to these dependencies still remains an issue. Dependency tracking issues are
also compounded by the scale of many software development projects leading to further information overload.

Social factors such as Cataldo et al.~\cite{Cataldo:2006:ICR} notion of socio-technical
congruence, have also been leveraged in indirect conflict tools~\cite{Kwan:2011:ESC, Begel:2010:CDE, Borici:2012:CHA}.
However, issues again of information overload, false positives, dependencies (in developer organizational structure), and scalability 
come up.

While these issues of information overload, false positives, dependencies, and scalability continue to come up
in most indirect conflict tools, only a handful of attempts have been made at rectifying these issues or
finding the root causes~\cite{Holmes:2010:CAR,Kim:2011:ESA}. 
In order to fully understand the root causes of information overload, false positives, and
scalability issues in regards to indirect conflicts, I will proceed by taking a step back and determine what events occur to
cause indirect conflicts, when they occur, and if conditions exist to provoke more of these events. By determining the root
causes of source code changes in indirect conflicts, we may be able to create indirect conflict tools which have filtered
monitoring in order to only detect those changes with a high likelihood of causing indirect conflicts.
I then set out to understand what mitigation strategies developers currently use as opposed to those created
by academia. Since developers have identified indirect conflicts as a major concern for themselves, but at
the same time are not using the tools put forth by academia, I wish to find what they use to mitigate indirect
conflicts. Through these findings, we can create tools which are similar to those already in use by developers in
the hopes of a higher adoption rate of tools produced by academia.
Finally, I look to find what can be accomplished moving forward with indirect conflicts
in both research and industry.

I restate the research goals of this thesis for ease of the reader: 

\begin{description}
  \item \textit{What events or conditions lead to indirect conflicts?}
  \item \textit{What mitigation techniques are used by developers in regards to indirect conflicts?}
  \item \textit{What do developers want from future indirect conflict tools?}
\end{description}

To answer these research questions, I performed a study (Section~\ref{study:exploration}) in which I interviewed 19 developers from across 12 commercial and open source projects, 
followed by a confirmatory survey of 78 developers, and 5 confirmatory interviews. Based on the findings (to be seen) to the aforementioned research questions, 
I also performed a secondary complimentary study of software evolutionary trends (Section~\ref{study:apie}) to solidify
developer opinion and provide a starting point for the future development of indirect conflict tools.


\input chapters/3/sec-iced
\input chapters/3/sec-apie
