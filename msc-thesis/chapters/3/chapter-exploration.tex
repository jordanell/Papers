\startchapter{Exploring Indirect Conflicts}
\label{chapter:exploration}

Through the two previous studies, I have shown that developers linked indirectly in source code changes can statistically
related to software failures. In the attempts of mitigating these loses through added awareness, I implemented
an indirect conflict tool called \textit{Impact}. However, \textit{Impact} ultimately suffered from information
overload as seen in its evaluation which was caused by false positives and scalability of the tool.

While other indirect conflict tools have shown potential from developer studies, some of the same problems continue
to arise throughout most, if not all tools. The most prevalent issue is that of information
overload and false positives. Through case studies, developers have noted that current indirect conflict tools provide too many 
false positive results leading to information overload and the tool eventually being
ignored~\cite{Sarma:2007:TSA, Servant:2010:CPI}. A second primary issue is that of dependency identification and
tracking. Many different dependencies have been proposed and used in indirect conflict tools such as method 
invocation~\cite{Trainer:2005:BGT}, and class signatures~\cite{Sarma:2007:TSA} with varying success, but the 
identification of failure inducing changes, other than those which are already identifiable by other means such
as compilers, and unit tests, to these dependencies still remains an issue. Dependency tracking issues are
also compounded by the scale of many software development projects leading to further information overload.

Social factors such as Cataldo et al's.~\cite{Cataldo:2006:ICR} notion of socio-technical
congruence, have also been leveraged in indirect conflict tools~\cite{Kwan:2011:ESC, Begel:2010:CDE, Borici:2012:CHA}.
However, issues again of information overload, false positives, dependencies (in developer organizational structure), and scalability 
come up.

While these issues of information overload, false positives, dependencies, and scalability continue to come up
in most indirect conflict tools, only a handful of attempts have been made at rectifying these issues or
finding the root causes~\cite{Holmes:2010:CAR,Kim:2011:ESA}. 
In order to fully understand the root causes of information overload, false positives, and
scalability issues in regards to indirect conflicts in this study, I examine and determine what events occur to
cause indirect conflicts, when they occur, and if conditions exist to provoke more of these events. By determining the root
causes of source code changes in indirect conflicts, we may be able to create indirect conflict tools which have filtered
monitoring in order to only detect those changes with a high likelihood of causing indirect conflicts.
I then determine what mitigation strategies developers currently use as opposed to those created
by academia. Since developers have identified indirect conflicts as a major concern for themselves, but at
the same time are not using the tools put forth by academia, I wish to find what they use to mitigate indirect
conflicts. Through these findings, we can create tools which are similar to those already in use by developers in
the hopes of a higher adoption rate of tools produced by academia.
Finally, I examine what can be accomplished moving forward with indirect conflicts
in both research and industry.

To explore and answer the research goals listed above, I performed a study (Section~\ref{study:exploration}) in which I 
interviewed 19 developers from across 12 commercial and open source projects, 
followed by a confirmatory questionnaire of 78 developers, and 5 confirmatory interviews.

Based on some of the findings (to be seen in details in Section~\ref{study:exploration}) I performed a follow up study
which did not relate directly to the themes of this dissertation but helped strengthen the results found in Study 3.
Some results of Study 3 showed that indirect conflict tools should take into account a contextual setting of development
progression in software projects to better inform developers of potential indirect conflicts. In other words, an indirect
conflict tool should be able to tell what phase of development inside the development life cycle a project is currently
active in. In order to better explore and support this finding, I performed a complimentary study of software evolutionary trends (Section~\ref{study:apie}).
In Study 4, I perform a case study of 10 open source projects in order to study their source code change trends surrounding major release points
throughout their history. I studied 26 change trends quantitatively and 4 change trends qualitatively, and identified a core group of 9 change trends which occur
prominently at major release points of the projects studied. These change trends can provide context as to when indirect conflicts are more likely
to occur based on the findings from Study 3 as I found that indirect conflicts tend to be become less frequent near major release and more
frequent after a release or at the start of a new development cycle. The findings of this study can be applied over the lifetime of a project
to determine the proneness of indirect conflicts and thus aiding developers in dealing with indirect conflicts in their projects.


\input chapters/3/sec-iced
\input chapters/3/sec-apie
