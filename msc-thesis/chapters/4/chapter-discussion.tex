\startchapter{Discussion}
\label{chapter:discussion}

While each study presented in this dissertation is quite unique, this chapter will focus on the underlying themes and
results of all 4 studies. This chapter will address the usefulness of the results presented towards the
research community and
what those results mean for continued studies in future research as well as discuss how the results can be viewed
for larger over arching outcomes than those presented in results sections. 

This chapter will proceed in 2 subsections. The first section will address the 2 motivational studies and the lessons
learned from each as background information for the richer following studies. The second section will address the 2 large
studies found in Chapter 3 in a combined manner. Having reported the trends in industry (what, when, how, mitigation, and resolution)
from Study 3, the Study 3 discussion will focus these
results in relation to the outstanding issues of tool based solutions regarding indirect conflicts which are information overload,
false positives, and scalability. Since Study 4 was directly associated with developer opinion found in Study
3, the two discussions will be intertwined to better support each other.

\section{Motivating Studies Discussion}

\textbf{The Human Factor of Indirect Conflicts} \textit{An important component of indirect conflicts are the developers
themselves and how their notions of other's work is perceived across a project.}

As we have seen in Study 1, developers that are tied to source code objects can become a focal point of
indirect conflicts through the life of a project. For instance we can see from Table~\ref{tab:ratio} that
developers Daniel and Anne on Hibernate-ORM are always linked indirect conflicts when dependencies between
their source code objects change. This is an important observation to make when moving forward with indirect
conflicts.

What is really happening between these two developers is the real issue to consider. Daniel could have an
assumption about the way Anne's code works which causes Anne's changes to have negative impacts on Daniel's
own work. For instance, Daniel may expect a method of Anne's to have a special return case, which may be correct,
however when Anne changes that special case or removes it, Daniel's code can be negatively affected. In this extreme
result found in Study 1, further analysis showed that one central method in the software project was being changed
frequently and causing Daniel's code to be negatively affected in some way.

The question of how to prevent this type of negative impact is ultimately the goal of indirect conflicts. Study 1 has
shown that pairs of developers can often be a large component to that goal as well. (Impact from Study 2 was
created to use these pairs of developers in addressing the issue of indirect conflicts.) 

\textbf{Information Overload} \textit{My tool Impact, as well as many other indirect conflict tools, suffer from
information overload in their delivery to developers which is a key issue in preventing adoption and acceptance
of indirect conflict tools.}

As was previously stated, many indirect conflict tools end up suffering from information overload to developers
and other end users. Impact was initially created to take the insights from Study 1 and attempt to create
a new indirect conflict system based on developer interactions inside the code base which would potentially address
information overload. However, we now know from the results of Study 2, that Impact once again suffered defeat to information
overload based on the case study evaluation. Ultimately, this sense of information overload ends up causing
adoption of indirect conflict tools to fail which in turn causes some research components to have failed as
well.

From previous works, as well as from Study 2, we know that information overload is a large issue in indirect
conflicts. It has been found that a large number of dependencies in software caused by the nature large software
projects is a root issue in information overload~\cite{Sarma:2007:TSA,Servant:2010:CPI}. These large sizes in
dependencies are ultimately what cause information overload as dependencies cannot be evaluated as to their
relevance in a certain source code change with ease. In other words, we cannot determine which of the numerous
dependencies will fail (outside of compilation and testing failures) when source code is changed.

A potential solution derived from the evaluation of Impact comes from the ideas of Meyer~\cite{Meyer:1988} on 
``design by contract''. In this methodology, changes to method preconditions and postconditions are considered 
to be the most harmful. This includes adding conditions that must be met by both input and output parameters 
such as limitations to input and expected output. To achieve this level of source code analysis, the ideas of 
Fluri et al.~\cite{Fluri:2007:CDT} can be used on the previously generated ASTs for high granularity source code 
change extraction when determining if preconditions or postconditions have changed. While this solution does not 
wholly address the problems of information overload as previously stated through failures in dependencies, it does 
reduce the number of source code changes to  be analyzed which in turn will reduce the amount of information on 
the whole which is put forth by the indirect
conflict system. This solution however does also run the risk of missing more indirect conflicts as many conflicts
can occur outside of changes to method contracts. This solution represents my first ideas of fixing information overload in regards to
indirect conflicts and is again found in the later discussion of root causes of indirect conflicts found in
the next section.

\section{Indirect Conflict Exploration Discussion}

While the previous section discussed the main motivations for taking a step back on indirect conflicts in order to understand
better what can be done to improve developer work flow, this section will discuss exactly those steps back. From Studies 3 and
4 we will notice 3 main trends that have been discovered for indirect conflicts in both industry and in research. These 3 major
trends are: unwanted awareness, prevention versus cure, and the gap in software evolution analysis. This
section will also include a brief discussion regarding the implications, learned from studies 3 and 4, for both research and
for industry through tools.

\textbf{Unwanted Awareness} \textit{Developers tend to only care about the awareness of other's activities, if it causes
a negative influence on their own work. Developers only want to hear about another developer's actions if it forces them
to take some action because of it. This limited awareness is quite different from what literature suggests, which is larger
awareness about most, if not all actions, and it also suggests why developers see a high amount of what they believe to be
false positives, as the changes being reported are not causing them to take action. These reported non-actionable responses
lead to information overload and false positives.}

As we have seen, indirect conflicts are found to be quite a serious problem that occur frequently,
sporadically, and differ greatly from case to case. These conditions pose large issues for the creation of generalizable theories
or tools in regards to indirect conflicts. These underlying complexities are the probable cause of disinterest of software developers
to proposed or implemented tools as discussed in Section~\ref{sec:iced-related}. This inability to generalize is what I believe to be the
leading cause of information overload and false positives in the awareness system, causing developers to eventually ignore 
information being presented to them, rendering the system useless. These false positives are caused by a difference in what
developers consider to be false positives versus what awareness literature considers they are. This disjoint, as will
be seen, is caused by developers only considering events which cause some action to be taken on their part, to be true positives,
where as current awareness understanding would state any event which is related to an individual's 
work~\cite{Herbsleb06collaborationin}~\cite{Cataldo:2008:SCF}, to be true positives.
``You need a good understanding of what the code change is or else you will have a lot of false positives'' said one developer,
showing that not all changes around an object should be reported for awareness.

Developers have been found to have a great understanding of what they need to know about and more importantly what they don't in
their project awareness. To be able to fully understand a developer's awareness intuition, we can see from the results of this study that
developers only want awareness of an event if the event forces the developer to take some action. In a sense, developers
don't care about what they don't need to account for. ``We would want a high probability of the [reported] change being an issue'', meaning
that the developer only wants the awareness if the item will require action on their part to resolve the issue.
This sense of unwanted or limited awareness is crucial to understand why generalized
awareness techniques of difficult to generalize problems, such as indirect conflicts through generalized changes to 
classes~\cite{Sarma:2007:TSA}, or methods~\cite{Trainer:2005:BGT,Servant:2010:CPI}, often encounter difficulties of false
positives and information overload. 
Developers simply do not want to know about events which effect them, but require no action on their part. 

This unwanted awareness, or limited awareness, seems counter intuitive to current awareness understanding which would state that
being aware of all events in ones surrounding leads to higher productivity or other quality aspects. In theory this is correct,
but as it was found in practice, this is an incorrect assumption. 
In regards to this full awareness, one developer said ``There is no room for this in [our company], as tools are already in place for 
analysis of change[s] and code review takes care of the rest''. Since software developers have limits on their time, awareness of
all events surrounding a developer's work or project is not possible. \textit{Developers prefer to spend their limited time dealing with
the awareness of events which cause them to take some action (changing code, communication, etc) rather than simply being aware
of events occurring around their work which pose no direct threat to the consumption of their time.}

Of course, whether or not this unwanted awareness is the correct path for developers to take is an open question to consider. When
developers encounter a problem which could have been solved by having greater awareness of events which did not directly affect
them initially, we must consider the positive and negative influences of adding this, for now, unwanted awareness. A positive
influence of total, or near total
awareness of events at the developer level, would be the full understanding of a developer's work and environment which comes
with higher quality or understanding of the product. 
A negative influence would be that valuable developer time is spent understanding events which may not directly apply to themselves as opposed to 
producing more output of their own work. This balance between awareness and productivity is found to be a fine line in practice,
however, when given the choice, developers tend to, as previously stated, lean towards less awareness in order to, in their eyes,
be more productive.

\textbf{Prevention versus Cure}  \textit{Developers would rather spend their time on curative measures (fixing problems after they
arise) rather than preventative measures (preventing problems from happening). Developers see the task of fixing real problems
to be less of a burden than preventing potential problems because of available tools and their perceived notions of time management.
Focusing on cure is a direct response to the issue of scalability found in indirect conflict tools. Trying to prevent all conflicts 
is too large a problem to handle, while curing existing conflicts is manageable by a developer.}

We have seen through the results, that developers posses both tools and practices for the prevention,
detection, and resolution of indirect conflicts. We have also seen that through unwanted awareness, and the use of developer time,
that developers tend to prefer working on real issues that have already occurred as opposed to preventing issues of the future which
may never arise. This is neatly explained by the popular adage ``I work until something breaks'' taken by most developers. This mindset
is a clear example of developers taking a curative approach to software development opposed to a preventative approach.
Prevention here refers to taking precautionary steps to stop issues, indirect conflicts, form occurring in the first
place while cure refers to fixing issues as they arise which includes not attempting, or putting little attempt, into preventing
them in the first place.

Two out of the three identified prevention methods taken by developers are simple blanket risk mitigation strategies
accomplished essentially by
not changing code (design by contract, and add and deprecate), while the third is simply developer experience and knowledge. Clearly,
developers are spending little to no time in prevention. Developers do however spend a large amount of time in detection and cure through the
writing of tests and the debugging of issues. In fact, most improvements mentioned by developers in regard to dealing with indirect
conflicts occurred at either the detection or cure levels. Obviously, developers either prefer to spend their time in curative
measures, or do not posses the proper tools to take better action in the preventative stages. ``You're reaction time is much more crucial''
said one developer in that resolution tools are believed to be or larger importance as once issues have occurred, it doesn't
matter what prevention was taken, the issue must simply be resolves as quick as possible now.

The lack of prevention process and tools being used is believed to be due to 2 factors. The first being the identification of dependencies.
Even with an experienced system architect, identifying dependencies and notifying those involved is a daunting task 
which is ignored more than dealt with. The second being the knowledge of when a dependency will fail, also requires vast knowledge
of the product, more than anyone may have. This is compounded by the unique and sporadic nature of indirect conflicts. These 2 factors
are what ultimately have led to the amount of false positives and information overload seen in previous tools. The
abundance of detection and curative process and tools on the other hand shows once again developer's willingness to debug
real issues, that maybe even could have been prevented, rather than prevent the issues in the first place. With
this lack of prevention and abundance of curative measures, the question ultimately presented in this area is if 
curative measures are more productive than preventative measures.

Dromey~\cite{Dromey:2003} has raised the debate of prevention versus cure in software development and how difficult a problem it is
to measure. The pros and cons of prevention versus cure I have identified, are similar to those of unwanted awareness,
in that they result in a trade off 
of where time is spent and how productive each side of the argument truly is. If prevention can be shown to be more effective in that
it reduces the number of indirect conflicts or time spent debugging them compared to the time taken to prevent them, should we then
not be moving developers into a more preventative mindset; if curative can be shown to be more effective in that it takes less time
to fix real issues compare to preventing potential ones, should we not be putting more time into automatic debugging, such as
Zeller's Delta Debugging~\cite{Zeller:2002:ICC} or program slicing techniques~\cite{Weiser:1982:PUS}, or automatic test case creation.

A last interesting observation about current curative measures being taken in industry, is that developers view testing of software 
as curative, when one could easily make the argument that it is preventative in that most tests are written to pass originally and
are kept in place to ensure future changes do not cause issues. This mindset may come from the notion that writing tests is originally
thought of as part of normal code writing, meaning that developers see the extra task of test coverage as part of feature implementation.
However, if a test never fails, could it not be said that it is preventing changes from causing it to fail rather than detecting
when failures or indirect conflicts do occur? This may suggest that if, given the write tools, developers may no longer view preventative
measures as a ``burden'' and may be more inclined to take a more balanced preventative approach rather than mostly curative, as
``prevention is the goal'' was commonly said by developers.

This prevention versus cure discussion resides purely on the developer level and should be noted that it may not apply to system
designers, architects, or managerial stakeholders. 
From the project managers interviewed it was shown that they heavily favored planning and 
prevention (even though their prevention may be on more abstract levels than developers actually need) while leaving the curative
approaches to their developers. The prevention versus cure debate may have different outcomes depending on what level of abstraction
is being viewed in research.

%mature point, static analysis
\textbf{Gap in Software Evolution Analysis} \textit{Due to a lack of productive software analysis tools, caused by project infrastructures
and analysis unfriendly languages, contextual indications of indirect conflicts are often missed. This lack of analysis also causes
some software project configurations and software languages to be quite prone to indirect conflicts. This is yet another issue related
to scalability of existing indirect conflict tools.}

Indirect conflicts are more likely to occur,
from developer opinion, before a mature point is reached in the project's evolution. ``Once you get the API stable, people are 
better at communicating changes in regards to dependency concerns.''(This mature point may stem from a major release,
end of an iteration, a new feature being released or any point considered stable and reliable.) However, this context of a 
mature point, as well as any other
potential contextual attributes, are rarely identified or accounted for in regards to indirect conflicts. Developers have said
that different change types may occur at different rates throughout a project's life time and that this
may drastically effect the outcomes of indirect conflict tools or processes.

A deeper understanding of what context indirect conflicts occur in seems to be more of a success factor to indirect conflict
research than may have been previously thought. Static analysis may be useful in regards to indirect conflict context in that we
could identify trends surrounding the mature points of previous projects in order to give a better understanding of what it
means for a project to be pre or post mature point, which could affect the outcome of indirect conflict understanding.
These change trends are exactly what Study 4 set out to address.

Through Study 4, I showed that 9 critical change trends exist at a major release of a variety of open source projects. This
knowledge could help identify, as was previously stated, when indirect conflicts are more likely to occur in a project or not.
For example, the trend of adding test methods was found to occur more after a major release than before. This trend along with
the knowledge that indirect conflicts occur more at the start of a new development cycle can be used to predict times in development
where indirect conflicts are likely to occur automatically. An automated system may see that more test methods are being added than
before in the past 4 weeks or so and be able to alert developers to an instability in the code which in turn means an elevated
risk of indirect conflicts.

Similarly, the trend of adding private classes is seen to occur more so before a major release than after. This knowledge again
can be used in an automated system along with the fact that indirect conflicts are less likely to occur towards the end of a development
cycle. The automated system will be able to identify a stability in the project through private classes being added and alert developers
that the code is stabilizing and that large changes to the project should be avoided until the start of the next development cycle
in order to further reduce the risk of indirect conflicts.

This added level of context, through the notion of a mature point, 
would add to the prevention versus cure debate as previously discussed as well. Prevention may be a better
choice once a project has reached a mature point as the code base becomes more stable and source code changes become more dangerous
to the quality of the project. Curative measures may be a better choice before a mature point as code churn can be higher which
causes more bugs than can be prevented. These possibilities existing, it may be imperative to discover more of project stability
and the mature point in order to fully understand the nature of indirect conflicts and their context.

While this added sense of understanding is important to indirect conflicts, contextual attributes of a project's current progress
or measures of performance are often more synonymous with project management
rather than software developers themselves. An understanding of a project's evolution, a mature point being reached as an example,
may pose as a more useful tool for project management rather than developers. This more abstract tendency of indirect
conflict occurrences may add even more power to project management for evaluation of progress, code stability, and code reviews.

In regards to these contextual identifications in software projects, dependency identification and tracking is a key missing component of
indirect conflict analysis due to the weaknesses of static analysis. The gap in this identification comes from software and 
organizational structures of software teams. Between increased modularization (multiple sub projects or repositories), 
cross language dependencies, and languages which do not lend themselves to static analysis, static analysis tools have 
become quite limited in identifying and tracking dependencies where they were once strong. As software has become more
sophisticated over time, static analysis tools have gone from extremely useful to only occasionally useful. Since many
projects involve several languages, sub projects, and database schemas, static analysis has become cripplingly obsolete
in the industry of today. In order to move indirect conflicts and many other research areas forward, this gap of cross domain
static analysis must be filled.

As an example, relational database schemas are one of the highest sources of indirect
conflicts found in their projects, ``it breaks stuff all over the place'',
yet we know from Maule et al.~\cite{Maule:2008:IAD} that relational database schemas have 
been scarcely researched in terms of indirect conflicts. This falls in with my previous understanding of cross language
dependency tracking in that database schemas are independent of languages which may be on the receiving end of their output.

These increases of complexity in software products has left quite a gap in dependency identification and tracking which 
has lead to some of the deficiencies of indirect conflict research.


\subsection{Implication for Research}
\label{sec:implr}

\textbf{Prevention versus Cure} \textit{The largest implication for future research found in this dissertation is the need for continued study of the open question
``Which of prevention or cure is more effective for software development and indirect conflicts?''. This simple question will undoubtedly 
produce extremely complex answers.}

With software developers focusing their current efforts on curative measures, for
indirect conflicts, suggests
that while it may not be the most effective, it may be the easiest road for developers to take. This may be the path of
least resistance, but it may not be optimal. Software engineering as a whole should strive to answer this question or
provide more insight into possibilities, as its answer may determine many future actions taken by the research community.

Recommendations towards the study of prevention versus cure involve the examination of formal processes and tools used
by industry professionals with measurements of efficiency and effectiveness, similar to the work of
Tiwana~\cite{Tiwana:2008:ICD} in coordination tools. With these studies, we may find insight into the correct balance
of prevention versus cure, thus being able to increase developer productivity as well as identify more gaps in theory 
versus practice which may lead to improved tools or the abandoning of existing ones as was shown with UML~\cite{Petre:2013:UP}.

\textbf{Awareness Theory to Model Real World} \textit{the need to further characterize the
mismatch between awareness approaches and tools as developed in the research community, compared to awareness needs as perceived
by industry. We should push our awareness theories to model real development
environments where interruptions, time management, and full development life cycles are often large factors.}

As was identified in the previous discussion, while current understanding suggests that awareness of all events related to ones work
produces a more coherent understanding of a person's environment, developers find this to be overly time consuming
to the point where they only want to be aware of events which require action on their part. This difference, of what 
should be and what is, should be further understood to combat future potential failures in
tool development or theories attempted to be used in practice.

\subsection{Implication for Tools}
\label{sec:implt}

\textbf{Give Developers What They Want} \textit{Developers have their own notions of what they want, and how ``good''
that something is for their productivity. We should pay attention to what they want, but also be mindful that what
developers want may not always be good for them.}

The direct implication this research has on tool creation is that of current adoption among developers. As was stated,
developers are more keen to invest their time at the detection and cure / resolution stages of indirect conflicts. That being 
said, focus at these two stages for tool development will lead to larger adoption among developers.
It should be noted here that detection must come with an almost zero rate of false positives as
current tools (unit and integration and user testing), while they may not have 100\% recall, have almost 100\% precision.

\textbf{Stronger Source Code Analysis} \textit{With languages like JavaScript becoming popular, having projects with
multiple languages, and often having a database schema, we should focus on better techniques for static analysis based
on what industry standards are for languages and project structures.}

The more indirect implication of this research on tool creation is that of improving existing tools. While not all existing
tools are used for indirect conflicts alone (automatic debugging~\cite{Zeller:2005:WPF}, unit tests, etc), most of these tools
have a need for rapid expansion, according to developers, for dealing with indirect conflicts. The ability to have unit tests
automatically written for a given software object's contract, the ability to find a change  in an external project
which has broken a developer's own project, or any automation of the existing detection and resolution stages of indirect
conflicts are what developers currently seek. But of course, most of these implications rely on the improvement of 
static analysis tools.

These tool implications themselves imply the need of further development of static analysis tools. Static 
analysis lays at the heart of most if not all stages of indirect conflict research. We must be able to track and manage
software dependencies across the new landscape of software development that is multiple projects, repositories, and cross
language support. These improvements will allow the further development of both current and future indirect conflict
tools.

